%-------------------------------------------------------------------------------
%	SECTION TITLE
%-------------------------------------------------------------------------------
\cvsection{Projects}

\vspace*{-1.5mm}
%-------------------------------------------------------------------------------
%	CONTENT
%-------------------------------------------------------------------------------
\begin{cventries}


  % \project
  % {\href{https://resume.ist}{resume.ist}} % Organization
  % {Dec. 2018 - PRESENT} % Date(s)
  % {
  %   \begin{cvitems}
  %     \item {resume.ist is a platform where you can build your professional profile page }
  %     \item {.}
  %     \item {.}
  %   \end{cvitems}
  % }

  % \project
  % {\href{https://makino.games}{Makino Games}} % Organization
  % {Jun. 2019 - PRESENT} % Date(s)
  % {
  %   \begin{cvitems}
  %     \item {Makino Games is an interactive game that aims to teach core concepts of machine learning to middle school students.}
  %     \item {Currently developing the backend using NestJS and the game using Angular, PixiJS and Blockly.}
  %   \end{cvitems}
  % }


  % \project
  %   {\href{http://corpexx.com}{corpexx}} % Organization
  %   {Dec. 2018 - PRESENT} % Date(s)
  %   {
  %     \begin{cvitems}
  %       \item {corpexx is an online recruitment platform that aims to make personal job recommendations based on skill levels of the user, filter unqualified applicants by allowing companies to set threshold skill levels for a job post, and increase recruiter efficiency by showing most relevant experiences of applicants first.}
  %       \item {Currently developing a PWA using Angular, and backend with .NET Core and PostgreSQL.}
  %     \end{cvitems}
  %   }

  \project
    {\href{https://mahlas.co}{Mahlas (mahlas.co)}} % Organization
    {Apr. 2021 - PRESENT} % Date(s)
    {
      \begin{cvitems}
        \item {Mahlas, Turkish for "Pen name", is an invite-only space for your university. You can post anonymously, pseudonoymously, or with your name.}
        \item {Used React Native, Relay, and TypeScript to build the mobile app; implemented backend using NestJS, GraphQL, PostgreSQL, Prisma, and AWS.}
      \end{cvitems}
    }

  \project
    {Proland} % Organization
    {Mar. 2018 - Aug. 2018} % Date(s)
    {
      \begin{cvitems}
        \item {Proland is a machine learning solution to help farmers choose the best product for them using only a mobile application. It uses last years' crop specific harvest data and monthly weather conditions such us temperature, wind and precipitation to give yield estimates for crops.}
        \item {Competed in Imagine Cup, one of the biggest student technology competitions, and qualified for the World Finals as one of the top 50 teams.}
        \item {Used Python to extract weather data, created a cross platform application using Xamarin.Forms, implemented our backend using Firebase.}
      \end{cvitems}
    }
%---------------------------------------------------------
\end{cventries}
